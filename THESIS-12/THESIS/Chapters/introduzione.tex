\chapter{Introduzione}

\medskip

Nell`era in cui viviamo l`analisi dei dati riveste un ruolo cruciale in molti settori della nostra societ\`a.
In questo contesto il Machine Learning si \`e dimostrato un potente strumento per estrarre informazioni utili e conoscenza da dati complessi e voluminosi. 
Il \textit{Machine Learning (ML)} \`e un sottoinsieme dell'intelligenza artificiale (AI) che si occupa di creare sistemi che apprendono e migliorano le proprie performance in base ai dati che utilizzano. 

\begin{comment}
Esistono principalmente due categorie di modelli di apprendimento automatico: machine learning supervisionato e machine learning non supervisionato.
L`apprendimento supervisionato utilizza set di dati etichettati per addestrare gli algoritmi per classificare o prevedere i risultati in modo accurato.
L`apprendimento non supervisionato, utilizza gli algoritmi di machine learning per analizzare e organizzare in cluster i set di dati senza etichette. 
Nel presente lavoro di Tesi sono stati utilizzati dei modelli di machine learning supervisionato, avendo a disposizione dei set di dati etichettati per addestrare gli algoritmi.
\end{comment}

In particolare lo scopo del lavoro di Tesi \`e stato analizzare due approcci diversi all`apprendimento automatico: un approccio classico e un approccio time series. 
Una \textit{time series} \cite{time_series} pu\`o essere definita come un insieme di osservazioni ordinate rispetto al tempo. 
La differenza sostanziale tra i due approcci \`e che con un approccio time series si hanno informazioni non solo sull`istante di tempo corrente, ma anche su un numero di istanti di tempo precedenti scelto arbitrariamente.
Il fine ultimo del presente lavoro di Tesi \`e stato quello di comparare le performance di 4 modelli di machine learning in condizioni diverse: Logistic Regression, Linear Discriminant Analysis, Random Forest, XGBoost.
Le condizioni diverse sopracitate sono state date dall'addestramento dei modelli su set di dati differenti in base all'approccio utilizzato, classico o time series.
L`ipotesi che abbiamo voluto dimostrare \`e che un approccio time series migliori le performance dei modelli rispetto ad un approccio classico, avendo a dispozione informazioni anche su una finestra di istanti di tempo precedenti e non solo sull'istante di tempo corrente, quindi pi\`u dati disponibili durante l`apprendimento.

\vspace{1cm}

Il lavoro \`e organizzato nel seguente modo:
\begin{itemize}

  \item Capitolo 2: fornisce una panoramica su Machine Learning, Anomaly Detection e Time Series
  
  \item Capitolo 3: descrive le metodologie e le strategie utilizzate nel lavoro di Tesi

  \item Capitolo 4: analisi dei risultati ottenuti

  \item Capitolo 5: conclusioni della Tesi
  
\end{itemize}

\vspace{-0.5cm}
\vspace{-0.3cm}
